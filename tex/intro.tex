\section{Introduction}

Power/performance trade offs are well established compromises in the design of all embedded systems \cite{das2016reliability}. In both hardware and software domains, there is a great deal of formal and empirical knowledge which guides system architects towards optimal design time decisions, and myriad runtime operation modes (i.e., power saving modes) controlled locally or remotely \cite{senni2016non}. Approximate computing promises unprecedented power savings by introducing a trade off between power and another dimension: accuracy \cite{mittal2016survey}. For applications resiliant to innacurate computations \cite{xu2016approximate}, or where there isn't a single golden result \cite{venkataramani2016approximate}, approximate computing methods can improve traditional design strategies for power reduction: essential in the dark silicon era \cite{mitra2017power}.
\par Despite its promise, approximate computing is still an immature technology: a formal model of the impact of approximations on other design metrics does not yet exist \cite{venkataramani2015approximate}. Hence, most approximate computing applications require two premises to be implemented successfully: (a) adequate test data are available, to correctly model the accuracy impact of approximations \cite{yazdanbakhsh2017axbench}; and, (b) approximations are performed iteratively at design time, to meet the required power/accuracy goals, and remain static throughout deployment \cite{nepal2016automated}.
\par This is in stark contrast to performance/power trade offs, where well established benchmark suites offer near total coverage of application scenarios \cite{henning2000spec}: in approximate computing, test data that allows adequate modeling of accuracy is often unavailable. In performance/power trade-offs, systems can self-tune their operation based on load and run time parameters to dynamically adjust metrics \cite{isci2003runtime}. In approximate computing, approximations are static: mainly because there is no trusted method to determine if accuracy suffices, without access to ground truth \cite{chippa2013analysis}. In this paper, we tackle this problem:  adjusting the level of approximations at run time, for signal processing applications. Our hypothesis states that prior knowledge about processed data can guide built-in approximation engines, dynamically modifying the level of approximation whilst ensuring that accuracy suffices for the required task. We define prior knowledge as any heuristics or statistical assumptions about the data, which have been empirically or formally verified.


\par Specifically, this paper offers the following contributions:

\begin{itemize}
\item	We introduce the concept of prior knowledge-guided approximations. This represents a statistical measure of approximation impact, unlike test data-based empirical measures prevalent in the state of the art \cite{zhang2014approxit}. To be more specific, we use Kullback-Leibler (KL) divergence to evaluate if approximation error is within the acceptable bounds, changing approximation level accordingly at run-time.
\item	We introduce a model of run time approximations, which use prior knowledge to ensure that accuracy suffices, without access to ground truth, unlike iterative comparisons to ground truth prevalent in the state of the art \cite{mittal2016survey,yazdanbakhsh2017axbench}. Our approach dynamically modifies the level of approximation in function of KL divergence value.
\item 	We describe and evaluate a proof of concept of our approach, using an Extended Kalman Filter (EKF) for target tracking \cite{kulikov2016accurate}, where we have prior knowledge about the target's motion. Specifically, we assume that the object's motion statistics remain Gaussian.
\end{itemize}

\par The remainder of this paper is organized as follows: Section \ref{learning} describes a top level view of our methodology, explaining how prior knowledge can guide approximations dynamically. Section \ref{sec:EKF} describes a case study of our proposed method, where prior knowledge is used to dynamically adjust the approximations applied to an EKF for tracking. Section \ref{experiments} describes our experimental setup and the obtained results. Finally, Section \ref{conclusions} presents our conclusions and future work. 


\begin{figure}[tb]
  \centering
  \includegraphics[width=\columnwidth]{img/block_diagram.png}
  \caption{Block diagram}
  \label{fig:block_diagram}
\end{figure}
